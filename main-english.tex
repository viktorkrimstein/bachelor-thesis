% !TeX document-id = {af928648-7a45-486b-aa45-9066bfc9e040}
% !TeX spellcheck = en-US
% !TeX encoding = utf8
% !TeX program = pdflatex
% !BIB program = biber

\let\ifdeutsch\iffalse
\let\ifenglisch\iftrue
\input{pre-documentclass}
\documentclass[
a4paper,
twoside,
bibliography=totoc,
% dxtotoc,   %Index ins Inhaltsverzeichnis
% liststotoc, %List of X ins Inhaltsverzeichnis, mit liststotocnumbered werden die Abbildungsverzeichnisse nummeriert
headsepline,
cleardoublepage=empty,
parskip=half,
% draft um zu sehen, wo noch nachgebessert werden muss - wichtig, da Bindungskorrektur mit drin
draft=false
]{scrbook}
\input{config}

\usepackage[
title={Evaluation and Integration of an OPC UA Service into a Container-Based Cloud Platform},
author={Viktor Krimstein},
type=bachelor,
institute=iste, % or other institute names - or just a plain string using {Demo\\Demo...}
course={Software Engineering},
examiner={Prof.\ Dr.\ rer.\ nat.\ habil.\ Stefan Wagner},
supervisor={Dipl.-Ing.\ Wolfgang Fechner,\\Carsten Ellwein,\ M.Sc.},
startdate={November 22, 2017},
enddate={May 22, 2018}
]{scientific-thesis-cover}

\newacronym{api}{API}{Application Programming Interface}
\newacronym{cad}{CAD}{Computer-Aided Design}
\newacronym{cae}{CAE}{Computer-Aided Engineering}
\newacronym{cam}{CAM}{Computer-Aided Manufacturing}
\newacronym{capp}{CAPP}{Computer-Aided Process Planning}
\newacronym{cm}{CM}{Cloud Manufacturing}
\newacronym{cnc}{CNC}{Computer Numerical Control}
\newacronym{com}{COM/DCOM}{Component Object Model/Distributed Component Object Model}
\newacronym{cps}{CPS}{Cyberphysical System}
\newacronym{crm}{CRM}{Customer Relationship Management}
\newacronym{dama}{DAMA}{Design Anywhere, Manufacture Anywhere}
\newacronym{dds}{DDS}{Data Distribution Service}
\newacronym{erp}{ERP}{Enterprise Resource Planning}
\newacronym{hateoas}{HATEOAS}{Hypermedia As The Engine Of Application State}
\newacronym{http}{HTTP}{Hypertext Transfer Protocol}
\newacronym{iaas}{IaaS}{Infrastructure as a Service}
\newacronym{iot}{IoT}{Internet of Things}
\newacronym{isw}{ISW}{Institute for Control Engineering of Machine Tools and Manufacturing Units of the University of Stuttgart}
\newacronym{it}{IT}{Information Technology}
\newacronym{json}{JSON}{JavaScript Object Notation}
\newacronym{m2m}{M2M}{Machine-to-Machine}
\newacronym{mrp}{MRP}{Manufacturing Resource Planning}
\newacronym{nc}{NC}{Numerical Control}
\newacronym{nist}{NIST}{National Institute of Standards and Technology of the United States of America}
\newacronym{nosql}{NoSQL}{Not only SQL}
\newacronym{opc}{OPC}{Open Platform Communications}
\newacronym{opcua}{OPC UA}{Open Platform Communications Unified Architecture}
\newacronym{paas}{PaaS}{Platform as a Service}
\newacronym{plc}{PLC}{Programmable Logic Controller}
\newacronym{poc}{PoC}{Proof of Concept}
\newacronym{qos}{QoS}{Quality of Service}
\newacronym{rest}{REST}{Representational State Transfer}
\newacronym{rnp}{R'n'P}{Rent'n'Produce: Secure Cloud Service for the Commissioning and Control of Production Systems}
\newacronym{saas}{SaaS}{Software as a Service}
\newacronym{soa}{SOA}{Service-Oriented Architecture}
\newacronym{sql}{SQL}{Structured Query Language}
\newacronym{ui}{UI}{User Interface}
\newacronym{url}{URL}{Uniform Resource Locator}
\newacronym{xaas}{XaaS}{Everything is treated as a Service}

\makeindex

\begin{document}
	
	\iftex4ht
	\Configure{$}{\PicMath}{\EndPicMath}{}
	%$
	\Css{body {text-align:justify;}}
	\Configure{graphics*}
	{pdf}
	{\Needs{"convert \csname Gin@base\endcsname.pdf
			\csname Gin@base\endcsname.png"}%
		\Picture[pict]{\csname Gin@base\endcsname.png}%
	}
	\fi
	
	%Tipp von http://goemonx.blogspot.de/2012/01/pdflatex-ligaturen-und-copynpaste.html
	%siehe auch http://tex.stackexchange.com/questions/4397/make-ligatures-in-linux-libertine-copyable-and-searchable
	%
	%ONLY WORKS ON MiKTeX
	%On other systems, download glyphtounicode.tex from http://pdftex.sarovar.org/misc/
	%
	\input glyphtounicode.tex
	\pdfgentounicode=1
	
	%\VerbatimFootnotes %verbatim text in Fußnoten erlauben. Geht normalerweise nicht.
	
	\input{commands}
	\pagenumbering{arabic}
	\Titelblatt
	
	%Eigener Seitenstil fuer die Kurzfassung und das Inhaltsverzeichnis
	\deftripstyle{preamble}{}{}{}{}{}{\pagemark}
	%Doku zu deftripstyle: scrguide.pdf
	\pagestyle{preamble}
	\renewcommand*{\chapterpagestyle}{preamble}
	
	\section*{Abstract}
	
		In the context of Industry 4.0, services are brought to the forefront along with a higher degree of networking. 
		The vision of self-governing systems that exchange information and make decisions without human intervention is a primary goal, but not necessarily the focus of many projects. 
		
		The research project ``Rent'n'Produce: Secure cloud service for the commissioning and control of production system'' which is currently being processed at the Institute for Control Engineering of Machine Tools and Manufacturing Units of the University of Stuttgart, has the vision of advancing self-directed manufacturing systems. 
		Companies from different industries should be provided with a cloud-based platform that enables a highly flexible production order. 
		
		This bachelor thesis will focus on the integration of a real machine tool into the Rent'n'Produce Cloud Platform by using the Open Platform Communications Unified Architecture. 
		The required functionality of the integrated service should be encapsulated in a virtualization container. 
		This middleware is intended to provide access to configuration and status data of the production resource, to transfer numerical control programs to the machine tool, and to start or stop the part production.
		
	\newpage
	
	\section*{Kurzfassung}
	
		Im Rahmen von Industrie 4.0 werden Services einhergehend mit einem höheren Vernetzungsgrad in den Vordergrund gerückt. 
		Die Vision sich selbst steuernder Systeme, die ohne menschliches Zutun Informationen austauschen und Entscheidungen treffen, ist zwar übergeordnetes Ziel, steht aber nicht unbedingt im Fokus vieler Projekte.
		
		Das Forschungsprojekt \glqq Rent'n'Produce: Sicherer Cloudservice zur Beauftragung und Steuerung von Fertigungssystemen\grqq{} das aktuell am Institut für Steuerungstechnik der Werkzeugmaschinen und Fertigungseinrichtungen der Universität Stuttgart bearbeitet wird, hat die Vision sich selbst steuernder Fertigungssysteme voranzutreiben.
		Unternehmen unterschiedlicher Branchen soll eine cloudbasierte Plattform zu Verfügung gestellt werden, mit der eine hochflexible Fertigungsbeauftragung möglich ist, ebenso wie der administrative Zugriff auf Produktionsressourcen.
		
		Ziel dieser Arbeit ist die Anbindung einer realen Werkzeugmaschine basierend auf der Open Platform Communications Unified Architecture an Rent'n'Produce umzusetzen. 
		Die benötigte Funktionalität der Anbindung soll dabei in einem Virtualisierungscontainer gekapselt werden. 
		Diese Middleware soll den Zugriff auf Konfigurations- und Zustandsdaten der Produktionsressource, die Übertragung von Programmen für numerische Steuerungen auf die Maschine und das Starten wie Stoppen der Bearbeitung ermöglichen.
	
	\cleardoublepage
	
	% BEGIN: Verzeichnisse
	
	\iftex4ht
	\else
	\microtypesetup{protrusion=false}
	\fi
	
	%%%
	% Literaturverzeichnis ins TOC mit aufnehmen, aber nur wenn nichts anderes mehr hilft!
	% \addcontentsline{toc}{chapter}{Literaturverzeichnis}
	%
	% oder zB
	%\addcontentsline{toc}{section}{Abkürzungsverzeichnis}
	%
	%%%
	
	%Produce table of contents
	%
	%In case you have trouble with headings reaching into the page numbers, enable the following three lines.
	%Hint by http://golatex.de/inhaltsverzeichnis-schreibt-ueber-rand-t3106.html
	%
	%\makeatletter
	%\renewcommand{\@pnumwidth}{2em}
	%\makeatother
	%
	\tableofcontents
	
	% Bei einem ungünstigen Seitenumbruch im Inhaltsverzeichnis, kann dieser mit
	% \addtocontents{toc}{\protect\newpage}
	% an der passenden Stelle im Fließtext erzwungen werden.
	
	\listoffigures
	\listoftables
	
	% Wird nur bei Verwendung von der lstlisting-Umgebung mit dem "caption"-Parameter benoetigt
	% \lstlistoflistings 
	% ansonsten:
	% \ifdeutsch
	% \listof{Listing}{Verzeichnis der Listings}
	% \else
	% \listof{Listing}{List of Listings}
	% \fi
	
	% mittels \newfloat wurde die Algorithmus-Gleitumgebung definiert.
	% Mit folgendem Befehl werden alle floats dieses Typs ausgegeben
	% \ifdeutsch
	% \listof{Algorithmus}{Verzeichnis der Algorithmen}
	% \else
	% \listof{Algorithmus}{List of Algorithms}
	% \fi
	% \listofalgorithms %Ist nur für Algorithmen, die mittels \begin{algorithm} umschlossen werden, nötig
	
	% Abkürzungsverzeichnis
	\printnoidxglossaries
	
	\iftex4ht
	\else
	% Optischen Randausgleich und Grauwertkorrektur wieder aktivieren
	\microtypesetup{protrusion=true}
	\fi
	
	% END: Verzeichnisse
	
	% Headline and footline
	\renewcommand*{\chapterpagestyle}{scrplain}
	\pagestyle{scrheadings}
	\pagestyle{scrheadings}
	\ihead[]{}
	\chead[]{}
	\ohead[]{\headmark}
	\cfoot[]{}
	\ofoot[\usekomafont{pagenumber}\thepage]{\usekomafont{pagenumber}\thepage}
	\ifoot[]{}

	%%%%%%%%%%%%%%%%%%%%%%%%%%%%%%%%%%%%%%%%%%%%%%%%%%%%%%%%%%%%%%%%%%%%%%%%%%%%%%
	%
	% Main content starts here
	%
	%%%%%%%%%%%%%%%%%%%%%%%%%%%%%%%%%%%%%%%%%%%%%%%%%%%%%%%%%%%%%%%%%%%%%%%%%%%%%%
	
	\chapter{Introduction} \label{ch:introduction}
	
		This chapter will give a brief introduction to this work.
		
		\section{Motivation} \label{sec:motivation}
		
			The motivation for this work explained.
			
		\section{Goal} \label{sec:goal}
		
			Description of the specific goal of this work. This chapter should also mark out, that the work specifis a proof-of-concept architecture and the implementaiton verifies the planed work.
			
	\chapter{Foundations}\label{ch:foundations}
	
		\section{Rent'n'Produce}\label{sec:rent_n_produce}
		
		\section{Cloud Computing}\label{sec:cloud_computing}
		
			Explanation of common cloud computing patterns and currently state of the technology respectively to the manufacturing and Internet of Things Industry.~\cite{mell2011nist}
		
		\section{Cloud Manufacturing}\label{sec:cloud_manufacturing}
		
			Brief introduction to cloud manufacturing. Including the general patterns of Cloud Computing and the use of them in the special calse of Cloud Manufacturing. Brief comparison / introduction to The Internet of Things.
		
		\section{OPC UA} \label{sec:opc_ua}
		
			Introduction to communication protocols in manufacturing. Reference to concurrent products on the market.
			
		\section{Software in the Production Environment}\label{subsec:software_in_the_production_environment}
		
			Patterns, Software, Virtualization, Cloud Computing, IoT and else.
		
	\chapter{State of the Art}\label{ch:state_of_the_art}
			
		\section{Spring Boot}\label{subsec:spring_boot}
		
		\section{Docker}\label{sec:docker}
		
		\section{Eclipse Milo}\label{sec:eclipse_milo}
		
		\section{Representional State Transfer}\label{sec:rest}
						
	\chapter{State of the Science} \label{ch:state_of_the_Science}
	
		Show the current stae of the science.
		
		\section{Cloud Manufacturing}\label{sec:industry_4}
		
			Summary over the following papers:
			
			\begin{itemize}
				
				\item Industry 4.0 changes through network~\cite{brettel2014virtualization}
				
				\item Automation~\cite{jazdi2014cyber}
				
				\item Multivendor system challenges~\cite{weyer2015towards}
				
				\item Cloud Computing Security~\cite{subashini2011survey}
				
				\item Security SCADA and production systems~\cite{igure2006security}
				
				\item CM Survey~\cite{he2015state}
				
				\item CM Service Models~\cite{li2010cloud}
				
				\item From Cloud Computing to Cloud Manufacturing~\cite{xu2012cloud}
				
				\item Communication protocolls~\cite{wollschlaeger2017future}
				
				\item Industrial Internet of Things~\cite{jeschke2017industrial}
				
			\end{itemize}
		
		\section{Control Engineering in the Cloud}\label{sec:control_engineering_in_the_cloud}
		
		\section{Machine-to-Machine Communication}\label{sec:machine_to_machine_communication}
		
			Describe the field of M2M by referencing the purposes of OPC UA and other currently available standards. Papers:
			
			\begin{itemize}
				
				\item Text
				
			\end{itemize}
				
	\chapter{Concept} \label{ch:concept}
	
		Explain all concepts and ideas for the solution
		
		\section{Requirements} \label{sec:requirements}
		
			Brief introduction to the requirements and needs, this work aims for.
		
			\subsection{Infrastructure} \label{subsec:infrastructure}
			
				Description of the infrastructure / platform on which, the service should be implemented on.
			
			\subsection{Architecture} \label{subsec:architecture}
			
				Brief overview of the current architecture, the service should be integrated on.
			
			\subsection{Technology} \label{subsec:technology}
			
				Overview of the techology stack used.
			
				
			\subsection{Mandatory Features} \label{subsec:mandatory}
				
				Description of the Mandatory Use Cases.
				
			\subsection{Optional Use-Cases} \label{subsec:optional}	
				
				Description of the optional Use Cases.
				
		\section{Approach} \label{sec:approach}
		
			Introduction to the approach chapter. Point out the proof of concept idea and that it should verify an Open Source service cloud manufacturing approach for further use.
		
			\subsection{Integration} \label{subsec:integration}
			
				Docker, already available R'n'P Platform, etc.
				
			\subsection{Target Architecture} \label{subsec:target_architecture}
			
				Component Diagram etc.
			
			\subsection{Services} \label{subsec:services}
			
				Service Component Diagram taken by the specification, describing which use-case should be handled by which service.
				
			\subsection{Workflow Concept} \label{subsec:workflow_concept}
			
				Sequence Diagram describing the whole workflow of the service and the platform from producer and consumer view.
				
			\subsection{Machine Tool Integration} \label{subsec:machine_tool_intergation}
			
				Description of evaluated NC and PLC decision. Description should reference the general OPC UA file type and lead, to a generic implementation independant of the underlying control.
				
			\subsection{Security} \label{subsec:security}
			
				Details on the security patterns of cloud computing implemented and the patterns of OPC UA Security used in this work.
				
	\chapter{Discusion} \label{ch:discusion}
	
		Discusion how well the actual implementation meets the defined concept goals.
		
		\subsection{Integration Results}\label{subsec:integration_results}
		
			Description of how well the service could be integrated.
			
		\subsection{Use-Case Implementation}\label{subsec:use_cases_implementation}
		
			Discussion and listing of the use-cases and requirements implemented.
			
		\subsection{Limitations}\label{subsec:limitations}
		
			Disscussion of the limitations and difficulities discovered as part of this work.
		
	\chapter{Conclusion and Outlook} \label{ch:conclusion_and_outlook}
	
		Conclusion and Outlook of the work.
		
	\clearpage
	
	
	%%%%%% BEGIN DEMO CONTENT %%%%%%
	
%	\chapter{Introduction}
%	
%	This thesis tarts with \cref{chap:k2}.
%	
%	We can also typeset \verb|<text>verbatim text</text>|.
%	Backticks are also rendered correctly: \verb|`words in backticks`|.
%	
%	\chapter{Chapter Two}
%	\label{chap:k2}
%	
%	LaTeX hints are provided in \cref{chap:latexhints}.
%	
%	\blinddocument
%	
%	
%	\chapter{Conclusion and Outlook}
%	\label{chap:zusfas}
%	
%	\section*{Outlook}
%	
%	\appendix
%	\input{latexhints-english}
%	
%	\clearpage
	
	%%%%%% END DEMO CONTENR %%%%%%
	
	% \printindex
	
	\printbibliography
	
	All links were last followed on \today.
	
	\pagestyle{empty}
	\renewcommand*{\chapterpagestyle}{empty}
	\Versicherung
\end{document}
